\documentclass[11pt,]{article}
\usepackage[left=1in,top=1in,right=1in,bottom=1in]{geometry}
\newcommand*{\authorfont}{\fontfamily{phv}\selectfont}
\usepackage[]{mathpazo}


  \usepackage[T1]{fontenc}
  \usepackage[utf8]{inputenc}




\usepackage{abstract}
\renewcommand{\abstractname}{}    % clear the title
\renewcommand{\absnamepos}{empty} % originally center

\renewenvironment{abstract}
 {{%
    \setlength{\leftmargin}{0mm}
    \setlength{\rightmargin}{\leftmargin}%
  }%
  \relax}
 {\endlist}

\makeatletter
\def\@maketitle{%
  \newpage
%  \null
%  \vskip 2em%
%  \begin{center}%
  \let \footnote \thanks
    {\fontsize{18}{20}\selectfont\raggedright  \setlength{\parindent}{0pt} \@title \par}%
}
%\fi
\makeatother




\setcounter{secnumdepth}{0}




\title{A Pandoc Markdown Article Starter and
Template \thanks{Replication files are available on the author's Github
account (\url{http://github.com/svmiller}). \textbf{Current version}:
April 15, 2021; \textbf{Corresponding author}:
\href{mailto:svmille@clemson.edu}{\nolinkurl{svmille@clemson.edu}}.}  }
 



\author{\Large Steven V.
Miller\vspace{0.05in} \newline\normalsize\emph{Clemson University}  }


\date{}

\usepackage{titlesec}

\titleformat*{\section}{\normalsize\bfseries}
\titleformat*{\subsection}{\normalsize\itshape}
\titleformat*{\subsubsection}{\normalsize\itshape}
\titleformat*{\paragraph}{\normalsize\itshape}
\titleformat*{\subparagraph}{\normalsize\itshape}


\usepackage{natbib}
\bibliographystyle{apsr}
\usepackage[strings]{underscore} % protect underscores in most circumstances



\newtheorem{hypothesis}{Hypothesis}
\usepackage{setspace}


% set default figure placement to htbp
\makeatletter
\def\fps@figure{htbp}
\makeatother

\usepackage{hyperref}

% move the hyperref stuff down here, after header-includes, to allow for - \usepackage{hyperref}

\makeatletter
\@ifpackageloaded{hyperref}{}{%
\ifxetex
  \PassOptionsToPackage{hyphens}{url}\usepackage[setpagesize=false, % page size defined by xetex
              unicode=false, % unicode breaks when used with xetex
              xetex]{hyperref}
\else
  \PassOptionsToPackage{hyphens}{url}\usepackage[draft,unicode=true]{hyperref}
\fi
}

\@ifpackageloaded{color}{
    \PassOptionsToPackage{usenames,dvipsnames}{color}
}{%
    \usepackage[usenames,dvipsnames]{color}
}
\makeatother
\hypersetup{breaklinks=true,
            bookmarks=true,
            pdfauthor={Steven V. Miller (Clemson University)},
             pdfkeywords = {pandoc, r markdown, knitr},  
            pdftitle={A Pandoc Markdown Article Starter and Template},
            colorlinks=true,
            citecolor=blue,
            urlcolor=blue,
            linkcolor=magenta,
            pdfborder={0 0 0}}
\urlstyle{same}  % don't use monospace font for urls

% Add an option for endnotes. -----


% add tightlist ----------
\providecommand{\tightlist}{%
\setlength{\itemsep}{0pt}\setlength{\parskip}{0pt}}

% add some other packages ----------

% \usepackage{multicol}
% This should regulate where figures float
% See: https://tex.stackexchange.com/questions/2275/keeping-tables-figures-close-to-where-they-are-mentioned
\usepackage[section]{placeins}


\begin{document}
	
% \pagenumbering{arabic}% resets `page` counter to 1 
%    

% \maketitle

{% \usefont{T1}{pnc}{m}{n}
\setlength{\parindent}{0pt}
\thispagestyle{plain}
{\fontsize{18}{20}\selectfont\raggedright 
\maketitle  % title \par  

}

{
   \vskip 13.5pt\relax \normalsize\fontsize{11}{12} 
\textbf{\authorfont Steven V. Miller} \hskip 15pt \emph{\small Clemson
University}   

}

}







\begin{abstract}

    \hbox{\vrule height .2pt width 39.14pc}

    \vskip 8.5pt % \small 

\noindent This document provides an introduction to R Markdown, argues
for its benefits, and presents a sample manuscript template intended for
an academic audience. I include basic syntax to R Markdown and a minimal
working example of how the analysis itself can be conducted within R
with the \texttt{knitr} package.


\vskip 8.5pt \noindent \emph{Keywords}: pandoc, r markdown, knitr \par

    \hbox{\vrule height .2pt width 39.14pc}



\end{abstract}


\vskip -8.5pt


 % removetitleabstract

\noindent  

\hypertarget{introduction}{%
\section{Introduction}\label{introduction}}

Academic workflow, certainly in political science, is at a crossroads.
The \emph{American Journal of Political Science} (\emph{AJPS}) announced
a (my words)
\href{http://ajps.org/2015/03/26/the-ajps-replication-policy-innovations-and-revisions/}{``show
your work'' initiative} in which authors who are tentatively accepted
for publication at the journal must hand over the raw code and data that
produced the results shown in the manuscript. The editorial team at
\emph{AJPS} then reproduces the code from the manuscript. Pending
successful replication, the manuscript moves toward publication. The
\emph{AJPS} might be at the fore of this movement, and it could be the
most aggressive among political science journals, but other journals in
our field have signed the joint
\href{http://www.dartstatement.org/}{Data Access \& Research
Transparency} (DART) initiative. This, at a bare minimum, requires
uploading code from quantitatively-oriented published articles to
in-house directories hosted by the journal or to services like
\href{http://dataverse.org/}{Dataverse}.

Perhaps the greatest intrigue of R Markdown comes with the
\href{http://yihui.name/knitr/}{\texttt{knitr} package} provided by
\citet{xie2013ddrk}. In other words, the author can, if she chooses, do
the analysis in the Markdown document itself and compile/execute it in
R.





\newpage
\singlespacing 
\bibliography{master.bib}

\end{document}
